%%%%%%%%%%%%%%%%%%%%%%%%%%%%%%%%%%%%%%%%%%%%%%%%%%%%%%%%%%%%%%%%%%%%%%%%%%%%%%%%
%% File   : skoranne_homogenization.tex
%% Author : Sandeep Koranne (C) 2018
%%
%%%%%%%%%%%%%%%%%%%%%%%%%%%%%%%%%%%%%%%%%%%%%%%%%%%%%%%%%%%%%%%%%%%%%%%%%%%%%%%%
\documentclass{article}[12pt]
\usepackage{amssymb}
\usepackage{amsmath}
\usepackage{amsfonts}
\usepackage{amsthm}
\usepackage{a4wide}
\usepackage{times}
\usepackage{polynom}
\usepackage{graphicx}
\usepackage{enumerate}
\usepackage{subfigure}
\usepackage{geometry}
\usepackage{tikz}
\usepackage{listings}
\usepackage{algorithm}
\usepackage{calc}
\lstset{
  language={C++},
  basicstyle=\small,
  captionpos=b
}

\providecommand{\keywords}[1]{\textbf{\textit{Keywords---}} #1}

%\usetikzlibrary{calc}
%\usetikzlibrary{shapes}
%\tikzset{egrid/.style={draw,help lines}}
%\tikzset{mgrid/.style={draw,help lines,dashed}}
%\tikzset{epoint/.style={draw,circle,red,inner sep=2pt,fill}}
%\tikzset{mpoint/.style={draw,circle,blue,radius=3pt}}

\definecolor{dark}{RGB}{0,105,0}


%\newtheorem{lem}{Lemma}
\def\EE{\mathcal E}
\def\RR{\mathbb R}
\def\FF{\mathbb F}
\def\ZZ{\mathbb Z}
\def\CC{\mathbb C}
\def\Spec{\mathrm{Spec}}

\theoremstyle{plain}
\newtheorem*{main}{Main~Theorem}
\newtheorem{thm}{Theorem}
\newtheorem{cor}[thm]{Corollary}
\newtheorem{prop}[thm]{Proposition}
\newtheorem{lem}[thm]{Lemma}
      
%\setlength{\parindent}{0mm}  

\bibliographystyle{plain}


\def\NN{\mathbb N}
\def\QQ{\mathbb Q}

\begin{document}
\title{Homogenization Techniques for Calculating
  Effective Permittivity of Nano-composites}
\author{Sandeep Koranne}
\maketitle
\begin{abstract}
Recent work in the development of spherical nano-structures embedded in dielectric
has been shown to possess attractive optical properties which make them suitable for optical
computation,  a process which is based on light-light interaction and confinement.
Since experimental manufacture of such nano-materials is cost prohibitive and time consuming, a
numerical method which can analyze the optical behavior of nano-crystals with high accuracy and
efficiency is required.
For full system design of telecommunication systems, which have high computational
efficiency requirements, 
first-order accurate macro-models of nano-structures
are required. We present a homogenization technique which combines our previous work~\cite{koranne_optics}
in micro-model solution generation to analyze a periodic structure of identically
oriented Silicon nano-composites embedded in silicon-dioxide. We compare our approach
to the standard mean-field method to compute the effective permittivity of the
system. 
\end{abstract}
\keywords{
  Effective permittivity of silicon nano-composites,
  Nonlinear electromagnetic wave propagation, Maxwell's equations, FDTD method.}

\section{Introduction}
\label{sec:introduction}
The advent of optical computation in 
silicon photonics and nano-crystals has brought the problem of accurate
and efficient analysis of wave propagation in anisotropic nonlinear media to the forefront.
Recent work in the development of spherical nano-structures embedded in dielectric
has been shown to possess attractive optical properties which make them suitable for optical
computation,  a process which is based on light-light interaction and confinement.
Since experimental manufacture of such nano-materials is cost prohibitive and time consuming, a
numerical method which can analyze the optical behavior of nano-crystals with high accuracy and
efficiency is required. As Rukhlenko, \emph{et.~al.}~\cite{rukhlenko2012effective} have described
silicon nano-crystals (Si NC) have Kerr effect which is several orders of magnitude
stronger than undoped fused silica (SiO2). Therefore silicon photonics waveguides are being
integrated in photonics system. A schematic of identically oriented nano-composites is
shown in Figure~\ref{fig:nano-cylinder}. As can be imagined it is impractical (due to
computational requirements) to analyze the nonlinear effects of the waveguide by considering
each nano-sphere individually. Therefore, a macro-model of the Si NC which is atleast first-order
accurate, is needed. Previous approach to solving this problem include the effective-medium
approach as developed by Rukhlenko, \emph{et.~al.}~\cite{rukhlenko2012effective}.
In this paper we develop an alternative method based on homogenization of $\gamma$-periodic
structures. The sequel of the paper is as follows:~we first describe briefly the underlying
Maxwell equations with nonlinear susceptibility and polarization. Then we describe the FDTD
method we have used to solve the micro-cell problem. This is based on our auxiliary differential
equation (ADE) high-order non-linear FDTD method as described in our previous paper. A major
component of that paper is the development of the \emph{numerical flux} and its corresponding
conservation law as described in Section~\ref{section:micro-cell}. Having obtained the micro-cell
solution we perform standard homogenization theory  following closely the approach developed
by Abdulle and Weinan~E~\cite{abdulle2003finite}, with the matrix coefficient method
as described by Bensoussan, \emph{et.~al}~\cite{bensoussan2011asymptotic}.

The main contributions of our paper are the following:
\begin{enumerate}
\item{Micro-cell flux}:~we show an innovative use of the conservation of the numerical
  flux of the micro-cell problem. This numerical flux could be computed using nonlinear FDTD
  or other methods such as finite element methods,
\item{Macro model development}:~as the analytical construction of the matrix coefficient $A^0$
  in homogenization is complicated, we develop a numerical method for estimating the homogenization
  matrix using the micro-cell solution,
\item{Effective permittivity computation}:~we use the homogenization technique to compute
  the effective third-order susceptibility and permittivity of a parameterized system comprising
  of $(M\times N)$ identical spheres placed at a distance of $\gamma$
\end{enumerate}

\begin{figure}[htb]
\centering
\begin{tikzpicture}%[scale=0.8]
     \draw[color=black,step=0.25,dotted] (0,0)--(7,0);
     \draw[color=black,step=0.25,dotted] (0,1)--(7,1);
     \draw[color=black,step=0.25,dotted] (0,2)--(7,2);
     \draw[color=black,step=0.25,dotted] (0,3)--(7,3);
     \draw[color=black,step=0.25,dotted] (0,4)--(7,4);
     \draw[color=black,step=0.25,dotted] (0,5)--(7,5);
     
     \draw[color=black,step=0.25,dotted] (0,0)--(0,5);
     \draw[color=black,step=0.25,dotted] (1,0)--(1,5);
     \draw[color=black,step=0.25,dotted] (2,0)--(2,5);
     \draw[color=black,step=0.25,dotted] (3,0)--(3,5);
     \draw[color=black,step=0.25,dotted] (4,0)--(4,5);
     \draw[color=black,step=0.25,dotted] (5,0)--(5,5);
     \draw[color=black,step=0.25,dotted] (6,0)--(6,5);
     \draw[color=black,step=0.25,dotted] (7,0)--(7,5);

     %\draw[black] (2,2) circle(1);
     \draw[<->] (2,2)--(3,2);
     \draw (2.45,2) node[fill=white,below] {$\gamma$};
     \filldraw[black] (1.45,1.95) rectangle(1.55,2.05);
     \filldraw[black] (2.45,1.95) rectangle(2.55,2.05);
     \filldraw[black] (1.95,2.45) rectangle(2.05,2.55);
     \filldraw[black] (1.95,1.45) rectangle(2.05,1.55);     

     %\filldraw[black] (2,2) circle(1pt);
    \foreach \x/\y in { 1/,2/,3/,4/, 1/2,2/2,3/2,4/2, 1/3,2/3,3/3,4/3, 1/4,2/4,3/4,4/4 }
       \draw (\x,\y ) circle (0.2cm);
     
     \draw (2,0) node[fill=white,below] {$x_{1,i}$};
     \draw (0,2) node[fill=white,left] {$x_{2,j}$};            
\end{tikzpicture}
\caption{Example of nano-spheres embedded in dielectric.}
\label{fig:nano-cylinder}
\end{figure}

\section{Nonlinear Maxwell's Equations}
\label{section:nonlinear-maxwell}
We begin, as is customary, by noting Maxwell's equations in vector differential form for anisotropic media
in a given domain $\Omega$ for a time duration $[0,T]$:
\begin{eqnarray}
\frac{\partial \mathbf{D}}{\partial t} + \mathbf{J} & = & \nabla \times \mathbf{H} \label{eqn:mx1}\\
\frac{\partial \mathbf{B}}{\partial t} & = & - \nabla \times \mathbf{E} \label{eqn:mx2}\\
\nabla \cdot \mathbf{D} & = & \rho \label{eqn:mx3}\\
\nabla \cdot \mathbf{B} & = & 0 \label{eqn:mx4} 
\end{eqnarray}
Here, $\mathbf{E}$ represents the electric field. We use the $j\omega t$ convention for
representing time varying fields, therefore, the electric field is $\mathbf{E}=\mathbf{E_0}e^{j\omega t}$.
Similarly, $\mathbf{H}$ represents the magnetic field. $\mathbf{D}$ represents the electric
displacement, and for nonlinear medium, its interaction with $\mathbf{E}$ is critical and analyzed
below in more detail. The magnetic displacement is denoted by $\mathbf{B}$. The current density
is denoted $\mathbf{J}$. Ohm's law:
\[
\mathbf{J} = \sigma \mathbf{E}
\]
where $\sigma$ is the electric conductivity, relates the conduction current density to the electric
field. The continuity of charge equation can be written as a differential equation:
\[
\frac{\partial \rho}{\partial t} + \nabla \cdot \mathbf{J} = 0
\]
Finally, the constitutive relation between magnetic flux and magnetic field can be written as:
\[
\mathbf{B} = \mu_0 \mathbf{H} + \mathbf{M}
\]
where $\mu_0$ is the free space permeability.
In the present work, we assume non-magnetic materials, and assume that the domain has no charges and
no conduction currents.
For ease of reference we note the following vector calculus identities and relations:
\begin{equation}
\nabla \cdot \mathbf{V} = \Big( \hat{\mathbf{i}} \frac{\partial}{\partial x}
+ \hat{\mathbf{j}} \frac{\partial}{\partial y}
+ \hat{\mathbf{k}} \frac{\partial}{\partial z} \Big) \cdot
\Big( V_x\hat{\mathbf{i}} + V_y\hat{\mathbf{j}} + V_z\hat{\mathbf{k}}\Big) =
\frac{\partial V_x}{\partial x} + 
\frac{\partial V_y}{\partial y} + 
\frac{\partial V_z}{\partial z} 
\end{equation}
\begin{equation}
\nabla \times \mathbf{V} = \begin{vmatrix}
\hat{\mathbf{i}} & \hat{\mathbf{j}} & \hat{\mathbf{k}} \\
\frac{\partial}{\partial x} & \frac{\partial}{\partial y} & \frac{\partial}{\partial z} \\
V_x & V_y & V_z
\end{vmatrix}
= \hat{\mathbf{i}}\Big( \frac{\partial V_z}{\partial y} - \frac{\partial V_y}{\partial z}\Big)
+ \hat{\mathbf{j}}\Big( \frac{\partial V_x}{\partial z} - \frac{\partial V_z}{\partial x}\Big)
+ \hat{\mathbf{k}}\Big( \frac{\partial V_y}{\partial x} - \frac{\partial V_x}{\partial y}\Big)
\end{equation}
\begin{equation}
\nabla \times ( \nabla \times \mathbf{V} ) = \nabla(\nabla \cdot \mathbf{V}) - 
\nabla^2\mathbf{V} \label{eqn:curl_curl}
\end{equation}

As an example of the use of the above identities, we can derive wave equation
corresponding to Maxwell's
equations. Take the curl of Equation~\ref{eqn:mx2}:
\begin{equation}
\nabla \times \nabla \times \mathbf{E} = - \frac{\partial}{\partial t} \nabla \times \mathbf{B}
= -\mu_0 \frac{\partial}{\partial t} \nabla \times H
= -\mu_0 \frac{\partial}{\partial t} \frac{\partial \mathbf{D}}{\partial t}
= -\mu_0 \epsilon_0 \frac{\partial^2 \mathbf{E}}{\partial t^2} \label{eqn:wave}
\end{equation}
where we use Equation~\ref{eqn:mx1}.
 Using the curl identity given in
Equation~\ref{eqn:curl_curl} as well as the divergence relation given in Equation~\ref{eqn:mx3}
we have
\begin{equation}
\frac{1}{c^2} \frac{\partial^2 \mathbf{E}}{\partial t^2} - \nabla^2\mathbf{E} = 0 \label{eqn:wave1}
\end{equation}
where $c=\frac{1}{\sqrt{\mu_0\epsilon_0}}$ is the speed of light in free space.

\subsection{Nonlinear Polarization}
In a linear medium, the electric displacement $\mathbf{D}$ and the electric field $\mathbf{E}$
are related as:
\begin{equation}
\mathbf{D} = \epsilon_0\mathbf{E} + \mathbf{P} \label{eqn:electric_displacement}
\end{equation}
where $\mathbf{P}$ is the electric polarization and $\epsilon_0$ is the free space permittivity.
The theory of light propagation in dielectric solids is well explained in the book 
by Butcher and Cotter~\cite{butcher1991elements}. The material below is abstracted from the above sources.

\begin{equation}
\mathbf{P}(\omega,t) = \frac{ \frac{e^2}{m} \mathbf{E_0}e^{j\omega t}}{(\omega_0^2-\omega^2)+j\omega\frac{\gamma}{m}}
\label{eqn:polarization}
\end{equation}
The permittivity (witten as $\epsilon_0+\frac{\mathbf{P}}{\mathbf{E}}$) is therefore frequency dependent.
It should be noted that the permittivity is a complex number. If the medium has resonance at multiple frequencies
then the above equation for polarization has to be summed for each mode independently. We can expand
Equation~\ref{eqn:polarization} using a Taylor series in powers of $\mathbf{E}$, to get:
\begin{eqnarray}
\mathbf{P} & = & \mathbf{\chi}^{(1)}(\mathbf{E}) + \mathbf{\chi}^{(2)}(\mathbf{E}\otimes\mathbf{E}) + \mathbf{\chi}^{(3)}(\mathbf{E}\otimes\mathbf{E}\otimes\mathbf{E}) + \ldots \nonumber \\
           & = & \mathbf{P}^{(1)} + \mathbf{P}^{(2)} + \mathbf{P}^{(3)} + \ldots \label{eqn:nonlinear-polarization}
\end{eqnarray}
where $\mathbf{\chi}$ is a tensor coefficient, called the electric susceptibility. 
The book by Butcher and Cotter~\cite{butcher1991elements}~pp.~16, clearly
explains the use of tensor contraction in writing $\mathbf{P}^{(2)}$ as:
\begin{eqnarray}
\mathbf{P}^{(2)} & = & \epsilon_0 \int_{-\infty}^\infty d\tau_1 \int_{-\infty}^\infty d\tau_2 \mathbf{R}^{(2)} (t-\tau_1, t-\tau_2): \mathbf{E}(\tau_1) \mathbf{E}(\tau_2) \\
& = & \epsilon_0 \int_{-\infty}^\infty d\tau'_1 \int_{-\infty}^\infty d\tau'_2 \mathbf{R}^{(2)} (\tau'_1, \tau'_2): \mathbf{E}(t-\tau_1) \mathbf{E}(t-\tau_2) 
\end{eqnarray}
In the above equations, $\mathbf{R}^{(2)}$ is the quadratic polarization
response tensor of the medium, and the equations capture the time-invariance,
causality and \emph{intrinsic permutation symmetry} of the
polarization. Generalizations to $n=3$ are immediate, and it should be
noted that $\mathbf{R}$ then becomes a $(n+1)$ rank tensor.
Therefore we can now
write the electric displacement constitutive equation, Equation~\ref{eqn:electric_displacement} as:
\begin{equation}
\mathbf{D} = \mathbf{P}^{(1)}\mathbf{E} + \mathbf{P}^{(2)}\mathbf{E} + \mathbf{P}^{(3)}\mathbf{E} 
\label{eqn:electric_displacement_2}
\end{equation}
Next, we consider the dispersiveness of the medium; the linear polarization $\mathbf{P}^{(1)}$ can
be written as the sum of an instantenous response (corresponding to $\omega=\infty$) and a
dispersive response:
\begin{equation}
\mathbf{P}^{(1)} = \mathbf{P}^{(1)}_{\infty}(\mathbf{E}) + \mathbf{P}^{(1)}_{\mathrm{disp}}(\mathbf{E}) \label{eqn:electric_displacement_3}
\end{equation}

For notational simplicity we can split the above polarization into a linear as well
as a non-linear component
\begin{eqnarray}
  \mathbf{P} & = & \epsilon_\infty \mathbf{E} + \mathbf{\Phi} \\
  \mathbf{P^\mathrm{NL}} & = & a(1-\theta)||\mathbf{E}||^2\mathbf{E} + \mathbf{\Phi}^\mathrm{NL}
\end{eqnarray}
where $\mathbf{\Phi},\mathbf{\Phi}^\mathrm{NL}$ are the residual linear and non-linear
polarizations, and $a$ and $\theta$ are parameters which control the relative
strength and contribution of the instantaneous Kerr and the Raman response.
For comparison with the work of Rukhlenko, we only use the Kerr response in this
paper, but the proposed method is not restricted to it.

Using these the wave equation can be written as:
\begin{equation}
  \frac{1}{c^2} \frac{\partial^2 \mathbf{E}}{\partial t^2} - \nabla^2\mathbf{E} =
  \mu_0 \frac{\partial^2 \mathbf{P}}{\partial t^2} + \mu_0 \frac{\partial^2 \mathbf{P}^\mathrm{NL}}{\partial t^2}
  \label{eqn:wave2}
\end{equation}
Comparing Equation~\ref{eqn:wave1} and Equation~\ref{eqn:wave2}, we observe that
Equation~\ref{eqn:wave2} is a wave equation with a source term, and this is the underlying
reason for the nonlinear effects in electromagnetic wave propagation. The induced polarization
sets up a resonant electric field, which in turn acts as the source of a new wave. If certain
conditions (called the phase matching conditions) are satisfied, this process acts coherently
and a new harmonic (with a higher frequency than the original wave) is generated.


\section{Micro cell solution}
\label{section:micro-cell}
In our previous research paper~\cite{koranne_optics} we have described a high-order FDTD method for
wave propagation in nonlinear media.
\begin{equation}
E_{j-2} - 2E_{j-1} + E_{j}  = \frac{2h^2}{c^2}\Big[-v^2\epsilon_\infty E_j + - 4\beta v^2 E_j^2 - 9\gamma v^2 E_j^3\Big]
\end{equation}
We can write the above system of non-linear ODE in matrix form as $AE(t)=g(t)$, where $g(t)$ is a forcing function to account
for inflow and outflow boundary conditions at $z=0$ and $z=L=Mh$, where $\Omega=[0,L]$ is the computation domain. The matrix
system can be written as shown below using the notation
\begin{equation}
r(h,v) = 1-\frac{2h^2}{c^2}(-v^2\epsilon_\infty ) \label{eqn:rhv}
\end{equation}

\[
\begin{bmatrix}
r(h,v) & 0 & 0 & \ldots & \ldots & \vdots \\
-2 & r(h,v) & 0 & 0 & \ldots & \vdots \\
1 & -2 & r(h,v) & 0 & \ldots & \vdots \\
0 & 1 & -2 & r(h,v) & 0 & \vdots \\
\vdots & \vdots & \vdots & \vdots & \vdots & \vdots \\
0 & 0 & 0 & 1 & -2 & r(h,v) 
\end{bmatrix}
\begin{bmatrix}
\vdots \\
E_{j-2} \\
E_{j-1} \\
E_j \\
E_{j+1} \\
E_{j+2} \\
\vdots
\end{bmatrix} = 
\begin{bmatrix}
g_0(t)-\frac{2h^2 v^2}{c^2}( 4\beta E_0^2 + 9\gamma E_0^3) \\
\vdots\\
\vdots\\
\frac{-2h^2 v^2}{c^2}( 4\beta E_j^2 + 9\gamma E_j^3) \\
\vdots\\
\vdots\\
g_M(t)-\frac{2h^2 v^2}{c^2}( 4\beta E_M^2 + 9\gamma E_M^3)
\end{bmatrix}
\]

Using a centered difference formula for the time derivative we arrive at a method which is similar to Beam-Warming method.
\begin{equation}
\frac{E_j^{n+1}-2E_j^n+E_j^{n-1}}{\Delta t^2} = c^2\Big[ \frac{E_{j-2}^n - 2E_{j-1}^n + E_j^n}{2h^2}\Big]
\end{equation}
The conservation law for each $E_j(t)$ is a non-linear equation as shown below in Equation~\ref{eqn:rhv-conservation}.
\begin{equation}
E_{j-2}(t) - 2E_{j-1}(t) + E_j(t)\Big[1-\frac{2h^2}{c^2}(-v^2\epsilon_\infty ) 
+ \frac{2h^2 v^2}{c^2}( 4\beta E_j(t) + 9\gamma E_j^2(t))\Big] = 0 \label{eqn:rhv-conservation}
\end{equation}
The above can be converted to an explicit scheme, but Equation~\ref{eqn:rhv-conservation} has to be solved at each time step.
Therefore we get
\begin{eqnarray}
E_j^{n+1} & = & \frac{\Delta t^2 c^2}{2h^2}\Big[ E_{j-2}^n - 2E_{j-1}^n + E_j^n\Big] + 2E_j^n - E_j^{n-1} \nonumber \\
0 & = & E_{j-2}-2E_{j-1}+E_j^n\Big[ r(h,v) + \frac{2h^2 v^2}{c^2}( 4\beta E_j(t) + 9\gamma E_j^2(t))\Big]
\end{eqnarray}

From the above equation the numerical flux in the polarization can be calculated as below
\begin{eqnarray}
  D_j^{n+1} - \Phi_j^{n+1} & = & E_j^{n+1}[\epsilon_\infty + a(1-\theta)(E_j^{n+1})^2] \\
  \frac{H_j^{n+1/2}-H_j^{n-1/2}}{\Delta t} & = & - \frac{ (-E_{j+1}^n + \frac{1}{2} E_j^n) +
    (\frac{1}{6}E_{j+2}^n + \frac{1}{3}E_{j-1}^n) }{\Delta x} \\
  \frac{D_j^{n+1}-D_j^n}{\Delta t} & = & \frac{ (-H_{j-1}^{n+1/2}+\frac{1}{2}H_j^{n+1/2}) +
    (\frac{1}{6}H_{j-2}^{n+1/2} + \frac{1}{3}H_{j+1}^{n+1/2})  }{\Delta x} \label{eqn:numerical-flux}
  \end{eqnarray}

\section{Homogenization Technique}
\label{section:homogenization}
We refer to Abdulle~\cite{abdulle2003finite} and Bensoussan~\cite{bensoussan2011asymptotic} for
background material for this section, even though these references focus on the parabolic and
elliptic operators. Refer to Figure~\ref{fig:nano-cylinder}, where we have the periodic placement
of spheres with a distance periodicity of $\gamma$. Rather than treat each sphere individually
we can consider the multi-scale problem. The local electric field $\mathbf{E(r)}=(E_x,E_y,E_z)$ is
known everywhere inside the domain $\Omega$. Then the linear effective permittivity can be calculated
as:
\begin{equation}
  \epsilon_\mathrm{eff} = \frac{1}{V} \int_\Omega \epsilon(\mathbf{r})\Big(
  \frac{\mathbf{E(r)}}{\mathcal{E}}\Big)^2 dV
\end{equation}
where
\[
\mathcal{E}^2 = \mathcal{E}_x^2 + \mathcal{E}_y^2 + \mathcal{E}_z^2
\]
are the space-averaged electric field and components.
The space dependent permittivity is calculated based on the characteristic function $\chi_j(\mathbf{r})$,
where $\chi_j(\mathbf{r})=1$ when $\mathbf{r}$ is inside the $j$-th medium. Then
\[
\epsilon(\mathbf{r}) = \epsilon_1 \chi_1(\mathbf{r}) + \epsilon_2 \chi_2(\mathbf{r})
\]
where $\epsilon_1,\epsilon_2$ are the permittivities of silicon and silica, respectively.
For a volume fraction $f$ (of $V$), mean-field theory approximates the effective permittivity
as~\cite{rukhlenko2012effective}:
\begin{equation}
  \epsilon_\mathrm{eff}(\epsilon_1,\epsilon_2,f) = \frac{1}{4}
  \Big[ u + (u^2+8\epsilon_1\epsilon_2)^\frac{1}{2}\Big] \label{eqn:mft}
\end{equation}
where $u=(3f-1)\epsilon_1 + (2-3f)\epsilon_2$.

\subsection{Multiscale Formulation}
Let us denote the electromagnetic vector potential by $\mathcal{A}_\gamma(t,x)$, where
$\gamma$ is the period of the homogenization, i.e., $\mathcal{A}_\gamma(x) = \mathcal{A}(x/\gamma)$.
The electric field $E$ and magnetic
field are recovered by
\begin{eqnarray}
  E(t,x) = - \frac{\partial \mathcal{A}_\gamma}{\partial t}(t,x) \\
  B(t,x) = \mathrm{curl}\ \mathcal{A}_\gamma(t,x)
\end{eqnarray}
Using the wave equation, the homogenization theory implies that there is a weak solution
\[
E_\gamma \to E^0
\]
where $E^0$ is the solution to the homogenized problem
\[
\frac{1}{c^2} \frac{\partial^2 \mathbf{E^0}}{\partial t^2} - \nabla^2\mathbf{E^0} = 0
\]
and the electrostatic potential $\mathcal{A}_0$ is given by
\[
\mathcal{A}_{ij}^0 = \int_\Omega \Big(\mathcal{A}_{ij}(\mathbf{r}) + 
\mathcal{A}_{1,i}(\mathbf{r}) \frac{\partial \phi^j}{\partial x} +
\mathcal{A}_{2,j}(\mathbf{r}) \frac{\partial \phi^j}{\partial y}\Big) dV
\]
where $\Omega=(0,1)^2$, and we suppose $\mathcal{A}(\mathbf{r})$ is $\gamma$-periodic.
The $\phi^j(\mathbf{r})$ are the \emph{micro-cell} solutions we described in the
previous section.

\section{Proposed Algorithm}
\label{section:proposed_algorithm}
\begin{algorithm}

\end{algorithm}

\section{Implementation Details}
%% \begin{figure}
%% \begin{center}
%% %\includegraphics[width=10cm]{a.png}
%% \caption{Example of 1D FDTD Gaussian pulse propagation along the $z$-axis in non-dispersive media.}
%% \label{fig:1D-FDTD}
%% \end{center}
%% \end{figure}
\section{Experimental Results}

\section{Conclusion}


\bibliography{hom.bib}

\end{document}
